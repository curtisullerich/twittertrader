%File: formatting-instruction.tex
\documentclass[letterpaper]{article}
\usepackage{aaai}
\usepackage{times}
\usepackage{helvet}
\usepackage{courier}
\usepackage{hyperref}

\nocopyright
\frenchspacing
\pdfinfo{
/Title (Classification of Tweets by Company)
/Author (Curtis Ullerich, Daniel Stiner, Brandon Maxwell)}
\setcounter{secnumdepth}{0}  
\begin{document}

\title{Tweet Classification - Filtering \\ of Twitter for Company-relevant Tweets}
\author{
Curtis Ullerich, Daniel Stiner, Brandon Maxwell\\
Department of Computer Science and Engineering\\
Iowa State University
Ames, Iowa, USA\\
\{curtisu,stiner,bmaxwell\}@iastate.edu\\
}
\maketitle
\begin{abstract}
\begin{quote}
Twitter is a popular source for data mining due to its massive scale and inclusivity of current trends. As businesses seek to discover public opinion about their business or products, Twitter is one source for mining this information. Tweets present interesting classification challenges due to their relatively small number of features and irregular formatting.
\end{quote}
\end{abstract}

\noindent Tweet selection by keyword filtering often suffers from an high accuracy and low accuracy, or low accuracy and high recall. Using machine learning for text classification can improve this. We aim to survey a variety of text processing steps to increase accuracy of classification using a Naive Bayes classifier. We use a broad set of keywords with the goal of collecting a superset of the Tweets about a particular company (Apple, in this case) and train a binary classifier to filter out false positives.

\section{Introduction}

\subsection{Goals}

\begin{itemize}

\item Collect a corpus that represents a superset of tweets about Apple and label these instances as "apple" or "none"
\item Create preprocessing steps to maximize the accuracy of binary classification

\end{itemize}

\subsection{Hypotheses}
\item Using URL-replacement will improve disambiguation accuracy. 
\item adding features for emotican classes....
\item adding features for @references, retweets, and hashtags, will decrease the feature set while improving accuracy
\item selection of a proper tokenization regex will have profound effects on accuracy
\item Stopping and stemming....

\section{Related Work}



\section{System Breakdown}
\textbf{Corpus creation}\\
\textbf{Data aggregation and classification}\\
\textbf{Decision-making}\\

\subsection{Basic Approach}

We do all machine learning with the Mallet[2] library developed by the University of Massachusetts. 


\section{Results}

\section{Applications and Future Work}

\section{Contributions}

We have built several reusable Java components that extend the Mallet API for processig of data from Twitter. We created a TweetJsonIterator that accepts Twitter's JSON-formatted tweets and creates Mallet Instances containing their values for easy processing by Mallet's pipeline during model training. We have also created several Pipe implementations that can be reused or easily modified to suit a similar Twitter processing. These include FixEmoticons, Link2Title, Stemmer, SpellCheck, and TwitterFeatures. All code is available through our project website.

\section{ Acknowledgments}

[1] J. Bollen, H. Mao, and X.-J. Zeng. Twitter mood predicts the
stock market. Journal of Computational Science,
abs/1010.3003, 2010\\

[2] Mallett, University of Massachussetts Department of Computer Science\\

[3] Sanders Analytics (http://www.sananalytics.com/lab/twitter-sentiment/)

\end{document}
